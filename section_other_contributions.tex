% Other contributions


\begin{summary}[SUMMARY POINTS]
Our team engaged in multiple other activities supporting the WFIRST mission and the r that were
\begin{enumerate}
\item Each Main section will start with an executive summary in this box.
\item Summary point 1. These should be full sentences.
\item Summary point 2. These should be full sentences.
\item Summary point 3. These should be full sentences.
\item Summary point 4. These should be full sentences.
\end{enumerate}
\end{summary}

Our team engaged in multiple other activities supporting the WFIRST mission and the Project Office not listed before. We summarize them here.

\subsection{Focused \emph{Princeton Meetings} Participation}
%---------------------------------------------------------------

In addition to the active participation by the SIT leads to the 4 FSWG meetings that happened this year, team members participated to two focused so-called \emph{Princeton Meetings} meetings organized by Prof. Spergel in Princeton University and at the newly established Flatiron Institute in  New York, NY. These two meetings were focused respectively on planning the data processing need of the WFI given our collective experience with large astronomical datasets data processing, and ancillary science enabled by the HLS. Team members Lupton, Mandelbaum, Samushia, van der Linden represented our SIT at this meeting and presented their thoughts on these topics.

\subsection{Contributions to Evaluation of the WFIRST Supernova Program}
%------------------------------------------------------------------------

In addition to our work on the HLS cosmology surveys,
team members Olivier Dor\'e, Chris Hirata, David Spergel,
and David Weinberg all contributed to evaluating strategies and
requirements for the WFIRST supernova program, providing a
sounding board for the two supernova SITs and synthesizing
information for project management.  Some of these contributions
took place in sessions of the WFIRST FSWG meetings and some
of it in telecons and email exchanges with members of the
supernova teams.  Weinberg wrote an extensive referee report
for the Hounsell et al.\ paper (from the supernova SIT led by
Ryan Foley) on WFIRST supernova strategies.  Most importantly,
all four investigators participated in the ``supernova summit''
held at KIPAC in March 2017, reading background material from
the teams, participating in a day of presentations and discussions,
and writing a report for project management and the supernova SITs.


\subsection{Supported Postdoctoral Researchers and Graduate Students}
%--------------------------------------------------------------------

Our team has been very proactive at leveraging our collective involvement in other on-going large scale observational efforts such as the Dark Energy Survey (DES), the SUBARU Hyper-Suprime Cam (HSC) survey, but also the joint ESA/NASA Euclid mission and the Large Synoptic Survey Telescope Dark Energy Science Collaboration (LSST DESC) to create joint appointments and assemble a team of very strong postdoctoral researchers. In particular, the following researchers joined our team and are partially supported by out SIT:
\begin{itemize}
\item Ivano Baronchelli (Caltech/IPAC), working with Harry Teplitz;
\item Ami Choi (OSU), working with Chris Hirata and David Weinberg;
\item Shoubaneh Hemmati (IPAC/Caltech), working with Peter Capak;
\item Albert Izard (JPL), working with Alina Kiessling;
\item Niall MacCrann (OSU), working with Chris Hirata and David Weinberg;
\item Elena Massara (LBL), working with Shirley Ho;
\item Alex Merson (Caltech/IPAC), working with Yun Wang;
\item Hironao Miyatake (JPL/Caltech), working with Jason Rhodes and Tim Eifler;
\item Andres Plazas Malagon (JPL/Caltech), working with Jason Rhodes and Charles Shapiro;
\item Melanie Simet (UCR/JPL), working with Alina Kiessling;
\item Michael Troxel (OSU), working with Chris Hirata and David Weinberg;
\item Ying Zu (OSU), working with Chris Hirata and David Weinberg;
\item Chen He Heinrich (JPL), working with Olivier Dor\'e and Tim Eifler (starting in fall 2017);
\item Alice Pisani (Princeton), working with David Spergel (starting in fall 2017).
\end{itemize}

In addition several graduate students were partially supported by our SIT: fall, this list will be extended by
\begin{itemize}
\item XXX (OSU), working with Shirley Ho;
\item XXX (OSU), working with Rachel Mandelbaum;
\item YYY (OSU), working with Rachel Mandelbaum.
\end{itemize}

\subsection{Relevant Scientific Publications by Team Members}
%------------------------------------------------------------

Throughout our studies supporting the Project Office or the Mission, we published our results in scientific journals and made them available on the arXiv. The following \Oli{XXX} papers were published by our team members and motivated by our studies:

\begin{enumerate}
\item \citet{2017MNRAS.467..928G}, "Information Content of the Angular Multipoles of Redshift-Space Galaxy Bispectrum";
\item \citet{2016MNRAS.463.2708P}, "Optimal weights for measuring redshift space distortions in multitracer galaxy catalogues";
\item \citet{2016MNRAS.463..467K}, "Unbiased contaminant removal for 3D galaxy power spectrum measurements";
\item \citet{2016MNRAS.457..993P}, "Estimating the power spectrum covariance matrix with fewer mock samples";
\item \citet{2016PASP..128j4001P}, "The Effect of Detector Nonlinearity on WFIRST PSF Profiles for Weak Gravitational Lensing Measurements";
\item \citet{2017JInst..12C4009P}, "Nonlinearity and pixel shifting effects in HXRG infrared detectors";
\item \citet{2015MNRAS.449.2128K}, "Can we use weak lensing to measure total mass profiles of galaxies on 20 kpc scales?";
\item \citet{2017arXiv170406665M}, "The Complete Calibration of the Color-Redshift Relation (C3R2) Survey: Survey Overview and Data Release 1";
\item \citet{Schaan:2016ois}, "Looking through the same lens: shear calibration for LSST, Euclid \& WFIRST with stage 4 CMB lensing";
\item \citet{Chisari:2016xki}, "Multitracing Anisotropic Non-Gaussianity with Galaxy Shapes";
\end{enumerate}
