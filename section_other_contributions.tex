% Other contributions


\begin{summary}
Our SIT engaged in multiple activities supporting the WFIRST mission and the Project Office not discussed above. We summarize them here.
\begin{enumerate}
\item Our SIT actively participated to two workshops dedicated to the WFI;
\item Our SIT contributed to evaluate the WFIRST Supernova Program;
\item Our SIT partially supported 15 post-doctoral researchers and one graduate student;
\item Our SIT published 11 scientific papers supporting our studies or extending the scientific case of WFIRST;
\item SIT team members are co-leading and participating in the Tri-agency Cosmological Simulation Task Force (TACS);
\item SIT team members are co-leading large observational efforts supporting WFIRST science goals;
\item Our team delivered to the community multiple high-value products and softwares.
%--------------------------------------------------------------------------
\end{enumerate}
\end{summary}


\subsection{\emph{Princeton Meetings} Participation}
%---------------------------------------------------------------

In addition to the active participation by the SIT leads in the 4 FSWG meetings that happened this year, team members participated in two focused so-called \emph{Princeton Meetings} organized by Prof. Spergel at Princeton University and  the newly established Center for Computing Astrophysics (CCA) at the Flatiron Institute in  New York, NY. These two meetings were focused respectively on planning the data processing need of the WFI given our collective experience with large astronomical dataset data processing, and ancillary science enabled by the HLS. Team members Lupton, Mandelbaum, Samushia, and von der Linden represented our SIT at these meetings and presented their thoughts on these topics.

\subsection{Contributions to Evaluation of the WFIRST Supernova Program}
%------------------------------------------------------------------------

In addition to our work on the HLS cosmology surveys,
team members Olivier Dor\'e, Chris Hirata, David Spergel,
and David Weinberg all contributed to evaluating strategies and
requirements for the WFIRST supernova program, providing a
sounding board for the two supernova SITs and synthesizing
information for project management.  Some of these contributions
took place in sessions of the WFIRST FSWG meetings and some
 in telecons and email exchanges with members of the
supernova teams.  Weinberg wrote an extensive referee report
for the Hounsell et al.\ paper (from the supernova SIT led by
Ryan Foley) on WFIRST supernova strategies.  Most importantly,
all four investigators participated in the ``supernova summit''
held at KIPAC in March 2017, reading background material from
the teams, participating in a day of presentations and discussions,
and writing a report for project management and the supernova SITs.


\subsection{Support of Postdoctoral Researchers and Graduate Students}
%--------------------------------------------------------------------

Our team has been very proactive at leveraging our collective involvement in other on-going large scale observational efforts such as the Dark Energy Survey (DES), and the SUBARU Hyper-Suprime Cam (HSC) survey, in addition to the upcoming ESA/NASA Euclid mission and the Large Synoptic Survey Telescope Dark Energy Science Collaboration (LSST DESC) where we have been able to create joint appointments with our WFIRST SIT and have assembled a team of very strong postdoctoral researchers. In particular, the following researchers joined our team and are partially supported by out SIT:
\begin{itemize}
\item Ivano Baronchelli (Caltech/IPAC), working with Harry Teplitz on updating measurements of the H$\alpha$ luminosity function using HST data;
\item Ami Choi (OSU), working with Chris Hirata and David Weinberg on image simulations for the WL analysis;
\item Shoubaneh Hemmati (IPAC/Caltech), working with Peter Capak on defining the photometric redshift requirements of the WFIRST WL investigation;
\item Albert Izard (JPL), working with Alina Kiessling on cosmological simulations and the requirements driven by the need to accurately compute covariance matrices;
\item Niall MacCrann (OSU), working with Chris Hirata and David Weinberg on image simulations for the WL analysis;
\item Elena Massara (LBL), working with Shirley Ho on generating light cone simulations for the GRS survey;
\item Alex Merson (Caltech/IPAC), working with Yun Wang and Andrew Benson on generating light cone simulations for the GRS survey;
\item Hironao Miyatake (JPL/Caltech), working with Jason Rhodes and Tim Eifler on including modified gravity and observational systematic effects in the cosmological parameter likelihood;
\item Andres Plazas Malagon (JPL/Caltech), working with Jason Rhodes and Charles Shapiro on computing requirements on detector imperfections driven by the WL survey;
\item Melanie Simet (UCR/JPL), working with Alina Kiessling and Jason Rhodes on the WL analysis;
\item Michael Troxel (OSU), working with Chris Hirata and David Weinberg on image simulations for the WL analysis;
\item Ying Zu (OSU), working with Chris Hirata and David Weinberg on simulations for the galaxy cluster investigation;
\item Chen He Heinrich (JPL), working with Olivier Dor\'e and Tim Eifler (starting in fall 2017);
\item Alice Pisani (Princeton), working with David Spergel on void statistics for the GRS survey (starting in fall 2017);
\item Hao-Yi "Heidi" Wu (OSU), working with Chris Hirata and David Weinberg on simulations for the galaxy cluster investigation (starting in fall 2017).
\end{itemize}

In addition, Arun Kannawadi, a graduate student at CMU supervised by Rachel Mandelbaum has been working on image simulation and shape measurement analysis. He is partially supported by our SIT.

\subsection{Relevant Scientific Publications by Team Members}
%------------------------------------------------------------

In addition to our work supporting the Project Office, we published our results in scientific journals and made them available on the arXiv. The following 11 scientific papers were published by our team members and were motivated by our studies:

\begin{enumerate}
\item \citet{2017MNRAS.467..928G}, ``Information Content of the Angular Multipoles of Redshift-Space Galaxy Bispectrum";
\item \citet{2016MNRAS.463.2708P}, ``Optimal weights for measuring redshift space distortions in multitracer galaxy catalogues";
\item \citet{2016MNRAS.463..467K}, ``Unbiased contaminant removal for 3D galaxy power spectrum measurements";
\item \citet{2016MNRAS.457..993P}, ``Estimating the power spectrum covariance matrix with fewer mock samples";
\item \citet{2016PASP..128j4001P}, ``The Effect of Detector Nonlinearity on WFIRST PSF Profiles for Weak Gravitational Lensing Measurements";
\item \citet{2017JInst..12C4009P}, "Nonlinearity and pixel shifting effects in HXRG infrared detectors";
\item \citet{2015MNRAS.449.2128K}, ``Can we use weak lensing to measure total mass profiles of galaxies on 20 kpc scales?";
\item \citet{Masters2017}, ``The Complete Calibration of the Color-Redshift Relation (C3R2) Survey: Survey Overview and Data Release 1";
\item \citet{Schaan:2016ois}, ``Looking through the same lens: shear calibration for LSST, Euclid \& WFIRST with stage 4 CMB lensing";
\item \citet{Chisari:2016xki}, ``Multitracing Anisotropic Non-Gaussianity with Galaxy Shapes";
\item \citet{Merson2017}, ``Predicting H$\alpha$ emission line galaxy counts for future galaxy redshift surveys", currently under review by MNRAS.
\end{enumerate}

\subsection{Participation to the Tri-agency  Cosmological Simulation Task Force (TACS)}
%--------------------------------------------------------------------------
\label{sec:tacs}
The computational resources required to produce all of the cosmological simulations required for WFIRST are not yet well defined but they are known to be significant. WFIRST shares common goals with other upcoming cosmological surveys including LSST, Euclid, and DESI, so it makes sense to try to coordinate efforts between these projects. Team members have taken leadership positions in articulating and leading this effort.

Team members Kiessling and Heitmann have been asked to co-chair a Tri-Agency Cosmological Simulations (TACS) task force at the request of the Project leads from WFIRST, LSST, and Euclid. The co-chairs have formulated a charge for TACS that has been approved by the Tri-Agency (NASA, DOE, NSF), Tri-Project (WFIRST, LSST, Euclid) group (TAG). TAG has representation from each of the Agencies and Projects and is responsible for the coordination of joint data processing and cosmological simulations for the three Projects. JPL WFIRST Project Scientist Rhodes represents the WFIRST and Euclid Projects on the TAG, Program Scientist Benford represents NASA for WFIRST and WFI Adjutant (and SIT team member) Spergel represents the WFIRST FSWG. TACS consists of the two chairs, 9 task force members, and an advisory board of 7 people, with representation from WFIRST on the task force, advisory board, and from both of the co-chairs. Team member Mandelbaum is a member of the advisory board and team members Benson, Eifler, and Ho are task force members. EXPO SIT lead Robertson is also a member of the Advisory Board.

The primary goal of TACS is to investigate areas for coordination between WFIRST, LSST, Euclid, and DESI of supercomputing resources, supercomputing infrastructure, cosmological simulations, synthetic sky generation, systematics investigation, and workforce personnel. TACS will report their recommendations directly to TAG and the reports will be made public for transparency. The responsibility for implementing any recommendations or negotiating MOUs lies with the TAG. Work is currently underway in TACS and is anticipated to extend into FY18.

\subsection{Participation and Leadership Large Observational Program Supporting WFIRST Science Goals}
%--------------------

The WFIRST mission will provide an unprecedented survey (in depth and area) of the extragalactic sky at optical and near infrared wavelengths (0.5-2 $\mu$m).  However, data in the 3-5 $\mu$m wavelength range is an essential complement to measure stellar masses at $z>$3, probe the earliest sites of reionization, find the earliest quasars and structures, and see heavily obscured regions of galaxies and AGN.  Yet, besides Spitzer, no current or planned future mission is able to conduct deep yet wide area surveys at these wavelengths since JWST will be limited to a few square degrees of imaging over its lifetime. Co-I Capak is leading the Spitzer Legacy Survey (SLS) that will provide 20 square degrees of deep imaging. The Spitzer Legacy Survey enables new science and will improve the cosmological constraints provided by WFIRST mission.  The fields were chosen to be some of the most observable for missions at L2 (WFIRST, JWST, and Euclid), the North Ecliptic Pole (NEP) and the Chandra Deep Field South (CDFS).  The CDFS in particular is also the target of deep LSST data which will complement WFIRST for photometric redshifts.

Co-I Capak is also co-leading the Complete Calibration of the Color-Redshift Relation (C3R2) survey. As discussed in \S \ref{sec:wl_photoz}, a key goal of WFIRST is to measure the growth of structure with cosmic time from weak lensing analysis over large regions of the sky. Weak lensing cosmology will be challenging: in addition to highly accurate galaxy shape measurements, statistically robust and accurate photometric redshift (photo-$z$) estimates for billions of faint galaxies will be needed in order to reconstruct the three-dimensional matter distribution. The C3R2 survey is designed specifically to calibrate the empirical galaxy color-redshift relation to the Euclid, LSST and WFIRST. The C3R2 survey is obtaining multiplexed observations with Keck (DEIMOS, LRIS, and MOSFIRE), the Gran Telescopio Canarias (GTC; OSIRIS), and the Very Large Telescope (VLT; FORS2 and KMOS) of a targeted sample of galaxies most important for the redshift calibration. C3R2 focuses spectroscopic efforts on under-sampled regions of galaxy color space identified in previous work in order to minimize the number of spectroscopic redshifts needed to map the color-redshift relation to the required accuracy. Initial results include the 1283 high confidence redshifts obtained in the 2016A semester and released as Data Release 1 \citep{Masters:2107}.


\subsection{Community Deliverables}
%----------------------------------

Throughout our studies, our SIT aims to release codes, products, and simulated data sets, with the goals of building awareness of and broad support for the WFIRST dark energy program and inspiring the community to develop methods and carry out investigations that will maximize the cosmological return from WFIRST. This year, we released the following products available mostly from our SIT team webpage \href{http://www.wfirst-hls-cosmology.org/products/}{[Link]}.

\subsubsection{\CoLi\ Monte-Carlo Markov Chains}

Cosmological parameter MCMC chains corresponding to forecasts for the current survey of the WFIRST High Latitude Survey, combining weak-gravitational lensing (WL), cluster counts (CC), and redshift space distortions (GRS). These chains were computed using the CosmoLike software \citep{Krause2016}.

\paragraph{Multi-probe cosmology forecasts (including SN from both SIT SN teams) with realistic systematic budgets} The tightest constraints on cosmological models including cosmic acceleration, modified laws of gravity, and neutrino physics will come from a joint analysis of multiple cosmological probes. We forecasted constraints using “traditional” single probe analyses for clusters, BAO+RSD, and weak lensing as well as a multi-probe analysis that utilizes these and several other observables that can be extracted from the data: galaxy-galaxy lensing, photometric galaxy clustering, cluster weak lensing, spectroscopic galaxy clustering, SN from WFIRST (forecasts from David Rubin and Dan Scolnic from the two SIT SNe teams), SN from the existing Joint Lightcurve Analysis, existing Baryon Acoustic Oscillation information from BOSS, and CMB information from Planck. Since multi-probe analyses are highly constraining they impose tight requirements on systematics control. These systematics include uncertainties in the estimation of galaxy shapes and redshifts (photo-$z$, spec-$z$), cluster mass calibration, galaxy bias, and intrinsic alignment. Uncertainties due to baryonic effects (SN and AGN feedback, cooling) are not included. It is critical over the coming years to study these uncertainties and to develop the capability to control these systematics at the level of WFIRST multi-probe analyses.

\paragraph{WFIRST modified gravity studies} While cosmic acceleration models have a relatively established parameterization, this is not true for modified gravity theories. We released and showed our first forecasts of modified gravity scenarios, where deviations from Einstein GR are parameterized through $\mu$ and $\Sigma$ (please see \citet{Joyce:2016vqv} for an introduction and \citet{Simpson:2012ra} for details about this particular parametrization). Constraints on these modified gravity parameters combining weak gravitational lensing, galaxy-galaxy lensing, photometric galaxy clustering joint analysis for the nominal WFIRST survey (2,200 deg.$^2$) and for 2 extended survey scenarios (5,000 deg.$^2$ and 10,000 deg.$^2$, respectively) were obtained. We also studied the difference of this multi-probe case with a Weak Lensing only analysis.

\subsubsection{A WFIRST module has been added to the GalSim package}

GalSim is an open-source software for simulating images of astronomical objects (stars, galaxies) in a variety of ways. The bulk of the calculations are carried out in C++, and the user interface is in python. In addition, the code can operate directly on ``config" files, for those users who prefer not to work in python. The impetus for the software package was a weak lensing community data challenge, called \href{http://great3challenge.info/}{[GREAT3]}.

However, the code has numerous additional capabilities beyond those needed for the challenge, and has been useful for a number of projects that needed to simulate high-fidelity galaxy images with accurate sizes and shears. For details of the GalSim algorithms and code validation, please see \citet{Rowe:2015}.

We have now added a specific module to accurately simulate WFIRST images. The GalSim software package including the WFIRST module is available \href{https://github.com/GalSim-developers/GalSim}{[here]}.The development of this WFIRST module is included in version 1.4 and precedes our SIT. It already include inter-pixel capacitance, persistence, reciprocity and spider pattern. It will be periodically updated by SIT members as the WFIRST hardware and survey parameters are adjusted. We expect the next update including the latest layout of the FPA to be released as part of GalSim v1.5 before the end of August 2017.

If you mean the new changes that Rachel is making to match up with the Cycle 6 definitions, then this isn't quite done.  There is still a bit of confusion about exactly what the FPA layout is.  I think Chris is trying to nail down a couple of ambiguities from the project people.  But I think we expect this to be done and released as part of GalSim v1.5 within the next month at the latest.

If you mean the principle WFirst functionality including IPC, persistence, reciprocity, spider pattern, etc. that are in the GalSim WFirst module, these were part of v1.3, which was released on July 30, 2015.


\subsubsection{CANDELS Based Mock WFIRST and LSST Catalogs}

In order to accurately simulate WFIRST photometric redshifts critical for the WL investigation, we transformed the CANDELS Catalog to the LSST and WFIRST color system with an LSST cut applied (Peter Capak, Shoubaneh Hemmati). The catalog includes photometry estimates in LSST (u,g,r,i,z), WFIRST(Y,J,H,F184W) and k band (AB magnitudes) as well as photometric redshifts, FWHM of F160W (pixel, 1 pixel = 0.06 arcsec), and F160W AB magnitude from the original CANDELS catalogs. All five CANDELS fields (GOODS-S, GOODS-N, EGS, UDS and COSMOS) are used here which cover $\sim$ 0.2 deg.$^2$ CANDELS photometric catalogs are published in GOODS-S, COSMOS, UDS and EGS and will be published in GOODS-N (therefore not available to the public yet). In cases where the photometry in a WFIRST or LSST filter could not be measured using the neighboring filters in CANDELS due to non detections, the magnitude is set to 99.0 and the limiting magnitude in the closest band is recorded as the error.

This catalog has been released internally to our SIT for testing and validation first. It has been shared with other SITs and is publicly released on our SIT website.

\subsubsection{Interloper Fraction Calculator}

 We have also released a code that calculates the fraction of emission line galaxies which have emission lines others than H$\alpha$ or [OIII] appearing at the same wavelength observed by WFIRST and could thus lead to a wrong redshift estimate (the interloper fraction, Wong, Pullen \& Ho): a Python-based program that applies secondary line identification and photometric cuts to mock galaxy surveys, in order to simulate interloper identification.  We also have a module specifically designed to do WFIRST and predict interloper rates for WFIRST available  \href{https://github.com/kazewong/Intercut}{here} on GitHub \citep{Wong:2016eku}.
