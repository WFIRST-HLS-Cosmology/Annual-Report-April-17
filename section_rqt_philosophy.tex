%
% ** SECTION 4 **
%

The WFIRST science requirements process must 
%tie the cosmological constraints to statistical and systematic error budgets 
connect 
HLS hardware and software requirements to statistical and systematic error budgets and in turn to  cosmological constraints. While nominally a ``flow-down",  in practice this will be a iterative process as we optimize the science return within engineering  constraints. We will use different tools for each part of this process. 

At the highest level, we will use the \CoLi\ forecasting package (written 
by Co-I Eifler and collaborator Krause) % - mentioned in Exec  summary
to relate cosmological constraints to data set parameters (sky coverage, galaxy density) and parameterized descriptions of the systematic error budget. 
\CoLi\ is a multi-probe analysis and forecasting pipeline that is unique in its integrated ansatz of jointly modeling LSS probes and their correlated statistical 
%(including the dominant non-Gaussian terms) 
and systematic errors. Our team members have  applied it to SDSS WL data
\cite{Huff2014}, LSST forecasts \cite{Krause2015}, and DES data
\cite{Becker2015, DES2015}. \CoLi\ incorporates a full exploration of parameter space in place of the Fisher formalism, and it incorporates
a range of astrophysical (e.g., intrinsic alignments, nonlinear galaxy bias, 
baryonic effects)
and observational (e.g., shear calibration, photo-$z$ uncertainties) 
systematics. 
Eifler and Krause will actively maintain and
update \CoLi\ as part of our support of the FSWG.  Planned work includes adding new science extensions (spatial 
curvature and neutrino masses), WFIRST-specific systematic uncertainties, and the addition of higher order statistics to exploit the potential science return from the high number density of galaxies
in the imaging and spectroscopic surveys.

%CosmoLike represents a major leap beyond the previous generation of forecasting tools (e.g. those written for the SDT by Co-I Hirata), which will be used only as secondary checks.

WFIRST hardware capabilities (e.g., throughput, slew times) and observing strategy/time allocation determine the HLS's statistical power, whereas the ability to robustly constrain 
the instrument response model and astrophysical nuisance parameters determine the systematic errors. Statistical errors 
generally vary continuously as hardware parameters are changed, so the hardware requirements will reflect a joint assessment of science performance and engineering capabilities 
(including cost and risk). For the science assessment, we will build on our previous work on the Exposure Time Calculator (ETC) and 
operations simulations codes (both written by Co-I Hirata).  Both sets of tools
are fully automated and can treat the WL and GRS 
surveys with a common set of scripts. 
We will build an automated interface from these tools to 
\CoLi\ so that we can evaluate the science impacts of changes in WFIRST requirements (e.g., the static wavefront error budget). 
Our team will 
work in close coordination with project engineers to carry out a cost/benefit analysis of each such trade; Co-I Hirata has played this role extensively on the WFIRST SDTs 
and will be the lead for this task.

%
% Not sure that this paragraph is essential
%
% C.H. -- I think we need to discuss our approach to systematics here, as they are an important part of the error budget. I've tried to compress this as much as I can.
%
%Where practical, we will set the systematic error budget by requiring that their combination be smaller than the expected statistical errors. Two approaches are possible: 
%(i) to require the systematic in the observable (e.g. shear power spectrum) to be smaller than the statistical errors; and (ii) to require errors on e.g. $w$ to be 
%increased relative to statistical-only errors by some factor. Our preference is for (i) as it leads to simple and robust error budgeting: independent systematics can be 
%quadrature-summed, the requirement does not change if new cosmological parameters are added (e.g. neutrinos) or if external data sets are added, and no internal 
%consistency tests in the data are lost. In cases where such a tight systematic error budget becomes a cost or risk driver, we may in consultation with the Project 
%formulate a requirement using criterion (ii). Due to the potential for complex interactions among different systematics, all such instances will be explicitly tracked in 
%the forecasting machinery.
%
%Where practical, we will set the systematic error budget by requiring that their combination be smaller than the expected statistical errors. Our preferred approach is to 
%set systematics requirements in the observable space (e.g., shear power spectrum) rather than in terms of marginalized errors on $w$, since this leads to robust 
%error budgeting: independent systematics can be quadrature-summed, the requirement does not change if new cosmological parameters or external 
%data sets are added, and no internal consistency tests are lost. We will back away from this approach only if such a tight systematics budget becomes a 
%cost/risk driver. We will flow the systematics requirements down to the hardware (e.g., stability) and operations (e.g., survey 
%cross-linkage) using quantitative forecasting tools, for example predicting the covariance matrix of the flat field uncertainties. Co-I Hirata will be the lead 
%liaison with the Project team for these studies.
%
%Finally, we will develop the error budget validation methodology.  We will develop a list of systematic error control tests on the WL and GRS data and derive the requirements on the observing program.
%
%%and may require us to accept statistical power below the maximum theoretically achievable. We 
%%%will quantitatively assess the constraining power of the validation tests as we optimize the survey design and hardware requirements.
