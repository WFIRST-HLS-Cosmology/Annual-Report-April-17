%
% ** SECTION 8 **
% 

The coming decade will be an exciting time for cosmology. Before WFIRST launch,  major cosmological imaging surveys (KiDS, HSC, DES) and the DESI and PFS spectroscopic surveys will significantly advance our current understanding. WFIRST, Euclid and LSST  will then go further and survey the sky at optical and infrared wavelengths, the  James Webb Space Telescope (JWST) and the Extremely Large Telescopes (ELTs) will make very deep maps of the sky; eROSITA will survey the X-ray sky; CMB-S4 will make a deep map of the millimeter sky; and the Canadian Hydrogen Intensity Mapping Experiment (CHIME) and other radio surveys will map the large-scale distribution of H$\,${\sc i}.
%
%Our team will explore ways of maximizing the science return from these combined data sets. Our team members are playing significant roles in a number of these projects and have good ties to the other major surveys. 
%
A  goal of the pre-CDR activities will be to determine the analysis infrastructure and observations needed to achieve the full potential of WFIRST in combination with these  surveys. This is best done through a broad community effort that brings together scientists from these complementary projects. Our SIT team is extremely well-placed to do this, through team members' significant roles in a number of these projects and ties to the other major surveys. We will engage the community to identify and pursue the key areas where WFIRST and the concurrent projects will provide new opportunities to mitigate systematics and enhance the combined cosmological science return. 

Through organizing a number of open workshops (with some modest travel support budgeted for the team and community members) our SIT will incorporate the interplay between major planned surveys and WFIRST into the WFIRST strategy, to identify: (i) pre-launch observations, (ii) how these external data sets affect the WFIRST observing strategy (e.g., deep fields) and the instrument, and (iii) the software needed (to be built post-CDR) for combining these data sets.  

Themes we have identified for these multi-experiment workshops include: (i) Statistical methods for next generation cosmological analyses (ii) Strong lensing: synergies between ground and space (iii) Photometric redshift accuracy for next generation surveys (iv) LSST and WFIRST: data simulations and joint analyses to tackle systematics and reveal the dark sector.
 %(e) Community workshop on various themes, in conjunction with other  surveys if need be (strong lensing, synergies, 21cm). (f) Interplay with other major planned survey. 
%
%We plan to carry out a combination of activities to optimizing integrating these projects: (1) Identify the required pre-launch observations (2) Identify the needed software for combining these data sets.  Our goal would not be to write the software needed for these joint analyses pre-CDR but to identify how these external data sets effect the WFIRST observing strategy (e.g., deep fields) and instrument. (3) Engage the community in identifying and pursuing the new scientific opportunities in fundamental cosmology created by the combination of WFIRST  through organizing open workshops (and budgeted for modest travel supports for the team and the community). The themes we have identified for these workshops include: (a) Statistical methods for next generation cosmological analyses (b) Strong lensing: synergies between ground and space (c) Photometric redshifts for next generation surveys (d) LSST and WFIRST: simulations and joint analyses of data (e) Community workshop on various themes, in conjunction with other  
%  surveys if need be (strong lensing, synergies, 21cm). (f) Interplay with other major planned survey. 


Furthermore, through our team's international scientists in Canada and Japan, we plan to also engage these communities in WFIRST science.  If these  countries become partners, we hope that our international co-Investigators will receive support from their agencies and that they will serve as points of contact with their scientific communities and with the CHIME project (Smith) and the Subaru HSC/PFS project (Takada/Yoshida).

% If properly... the sum will be greater than its part  . Also Chime S4, HI nd X-rays

% We plan to organize open workshop of topics of relevance to... maximize  the cosmological return on the interaction of these datasets.

% Build on the team strong ties to all these collaborations. 

% LSST: Identified, what are the scientific rqt, what data format, . Identify pre-launch programs 

% Community engagement and external data-sets. Identify of pre-launch obsevations to optimize . Identify our rqt for this project. 

% Mention JWST. What we do before and after. Test on photo-$z$s. Test software to combine .

% WFIRST data-set will be transformational by itself (see e.g., \Oli{FigXX}). It also promises rich synergies with on-going and planned surveys. In the optical and near-infrared, WFIRST is scheduled to begin operations after the completion of ongoing major cosmological imaging surveys (KiDS, HSC, DES) and the planned spectroscopic survey, DESI. The LSST and ESA led Euclid surveys are  scheduled to overlap with WFIRST. These surveys present opportunities to dramatically improve the cosmological returns from WFIRST. Our team is uniquely positioned to enable them. 

% The biggest advantage from combining information from these missions is in the mitigation of systematic errors, especially via redshift information as highlighted recently by some of our team members \cite{Jain:2015cpa}. The systematic errors affecting the LSST, Euclid and WFIRST surveys individually  arise  from their incomplete wavelength coverage (creating photo-$z$ systematics), their differences in imaging resolution and blending (creating shear systematics), or from different biases in galaxy sample selections.  In most respects the  surveys are complementary in the sense that one survey reduces or nearly eliminates a systematic in the other.

% LSST is particularly relevant to our effort. Beyond being a major US community effort, the combination of LSST and WFIRST will be particularly mutually beneficial. LSST will use a large-aperture, wide-field, ground-based telescope  to perform repeated imaging observations of the entire southern hemisphere in six optical bands ($u$, $g$, $r$, $i$, $z$, $y$) to reach a limiting magnitude of about 27 in the $r$-band and comparable depth in the other five bands.  LSST plans to begin its 10-year survey in 2022. The optical color information from LSST will greatly aid the estimation of photo-$z$'s in the HLS. Conversely, the infrared data from WFIRST and its IFU spectra will aid in the estimation and calibration of LSST photo-$z$'s, a potential major source of systematics in WL tomography and cluster mass calibration.

% Our team has excellent connections to LSST and the other projects listed above. For example, several co-Is hold (or have held) leadership positions in LSST DESC, the team preparing for the cosmological analysis of LSST data (including plans for tackling systematics), in the DES survey, in the Euclid collaboration or in the DESI collaboration (see \S~\ref{sec:mgt} and \S~\ref{sec:bio}). We plan to use these connections to aggressively leverage the community effort in pipeline development, theoretical modeling, cosmological simulations, etc. for the mutual benefits of WFIRST and these other collaborations. 

% % the LSST DESC \Oli{Defined?} i Co-Is Bean, Jain, Jarvis, Ho, Kiesling, Mandelbaum and Krause (have) held leadership positions in DESC and will be actively involved in joint discussions with the DESC team. 

% To further engage the community in identifying and pursuing the new scientific opportunities created by WFIRST, we plan to organize open workshops (and budgeted for modest travel supports for the team and the community) on topical subjects of relevance to WFIRST. Beyond engaging the community in the development of WFIRST, it will also allow us to keep the WFIRST project attune with scientific developments throughout the years. Amongst potential topics of interest not discussed in the rest of this proposal, we identified the synergies with strong lensing and time delay information available from ground based surveys, cross-correlations with large scale low redshift 21 cm intensity mapping surveys (e.g., GBT-HIM, CHIME or HEREX \ref{XXX}) or statistical methods for large cosmological data-sets.

% %Beyond photo-$z$'s a number of additional gains can be made by using data from LSST and other surveys, in particular for galaxy clustering and strong lensing studies. 
% %There are a number of joint goals that can be efficiently pursued with the DESC team as well as other projects. 

% %Thus it is vital that plans are in place to exploit the synergies with complementary surveys. 

% %One of the  productive areas of collaboration is in the simulation studies needed to model systematics mitigation and forecast capabilities.

% %To pursue these specific synergies and other opportunities with the projects discussed above, we intend to host a number of community workshops. The themes we have identified for these workshops include: 
% %\begin{itemize}
% %\item Statistical methods for next generation cosmological analyses
% %\item Strong lensing: synergies between ground and space
% %\item Photometric redshifts for next generation surveys
% %\item LSST and WFIRST: simulations and joint analyses of data
% %\item Community workshop on various themes, in conjunction with other
%  % surveys if need be (strong lensing, synergies, 21cm).
% %\item Interplay with other major planned survey. List names and
%  % responsibilities.
% %\item science questions will evolve, technical challenge also. We will.
% %\item Strategy and activities for engaging the cosmological community
%  %in preparation for WFIRST. Be ready to use data-sets and conduct
%  %novel science. Think of the unexpected.
% %\end{itemize}

