%
% ** SECTION 1 **
%

\begin{summary}

Cosmic acceleration is the most surprising cosmological discovery in
many decades.  Even the least exotic explanation of this phenomenon
requires an energetically dominant component of the universe with
properties never previously seen in nature, pervading otherwise
empty space, with an energy density that is many orders of magnitude
lower than naive expectations. Testing and distinguishing among possible  explanations requires cosmological
measurements of extremely high precision that probe the full history of
cosmic expansion and structure growth.
This program is one of the defining objectives of the Wide-Field
Infrared Survey Telescope (WFIRST), as set forth in the {\it New Worlds, New Horizons}
report (NWNH) \cite{NWNH2010}.  The WFIRST mission, as described in the Science
Definition Team (SDT) reports \citep[hereafter SDT13 and SDT15 respectively]{Spergel2013, Spergel2015}, has the ability to improve these
measurements by $1-2$ orders of magnitude compared to the current
state of the art, while simultaneously extending their redshift grasp,
greatly improving control of systematic effects, and taking a unified
approach to multiple probes that provide complementary physical information
and cross-checks of cosmological results.

We described in this document the activities of the Science Investigation Team (SIT) \emph{Cosmology with the High Latitude Survey}.
This team was selected by NASA in December 2015 in order to address
the stringent challenges of the WFIRST dark energy (DE) program through the
Project's formulation phase.  This SIT has elected to address Galaxy Redshift Survey (GRS), Weak Lensing (WL) and Cluster Growth (CL) of the WFIRST Science Investigation Team (SIT) NASA Research Announcement (NRA) with a unified team, because the two investigations are tightly linked at both
the technical level and the theoretical modeling level. Our team thus fully embrace
the fact that the imaging and spectroscopic elements of the High Latitude Survey
(HLS) will be realized as an integrated observing program, and they jointly
impose requirements on instrument and telescope performance, operations, and
data transfer. We also naturally acknowledge that the methods for simulating and interpreting weak lensing and
galaxy clustering observations largely overlap. Many members of our team
have expertise in both areas.
%The WFIRST supernova cosmology program (investigation B) is more
%distinct in its methods and requirements, so it is feasible to
%integrate the supernova and HLS investigations at the level of the
%Formulation Science Working Group (FSWG).

In our proposal, we structured our planning around the series of deliverables
described in \S \ref{sec:deliverables}. We will present in this detailed report our progress on
these deliverables and illustrate that we either reached or exceeded our proposed
expected milestones.

Because development of requirements is at the core of our proposed
investigation, we present some broad aspects of our strategy in \S
\ref{sec:reqt_philosophy} before giving a summary of the High Latitude Imaging
Survey (HLIS) and of the HLS Spectroscopic Survey (HLSS) as we formulated them
to support the WFIRST Project Office in \S \ref{sec:wl_gal-clusters} and \S
\ref{sec:gc}. We present our revised cosmological forecasts and associated trade studies in
\S \ref{sec:forecast}. We also address questions of survey operations and
optimization in \S \ref{sec:operation}, our plans for broad community engagement
in \S \ref{sec:engagement} and discuss in \S \ref{sec:other_contributions} the
other ways in which our SIT supported the WFIRST mission.
%We conclude with a quick outline of the expected milestones for the coming year in \S
%\ref{sec:workplan}.
\end{summary}

%\addcontentsline{toc}{section}{References}
%We highlighted in bold the team members in the following bibliography.


%The team PI, Olivier Dor\'e, is a cosmologist with broad expertise in
%cosmic microwave background and large scale structure (LSS) studies. He brings
%extensive experience with complex data analysis (e.g., the Wilkinson
%Microwave Anisotropy Probe (WMAP), Planck) and mission design (e.g.,
%Joint Dark Energy Mission (JDEM) Destiny and the
%SMall EXplorer (SMEX) concept SPHEREx currently
%under Phase A study, for which he is the Project Scientist). Yun Wang
%and Chris Hirata will serve as Lead Co-Investigators for topics A and C,
%respectively and David Weinberg will serve as Lead for sub-topic
%``Cluster growth'' within topic C.  Many members of our team have been involved with the
%design and requirements of a dark energy space mission for a decade or
%more, including the Co-Chair (Spergel) and four additional members
%(Hirata, Hudson, Wang, Weinberg) of the 2013-2015 WFIRST-AFTA SDTs.  Our team
%includes authors of the two most comprehensive reviews of observational
%methods for probing dark energy \cite{Wang2010, Weinberg2013} and the
%Chair and Vice-Chair (Spergel, Weinberg) of the Astro2010 Science
%Frontier Panel on Cosmology and Fundamental Physics, whose report played
%a central role in the NWNH recommendation of WFIRST as the highest
%priority large space-based program.  Our team of Co-Is includes world
%leading experts on image processing and weak lensing (Eifler, Jain,
%Jarvis, Kiessling, Lupton, Hirata, Mandelbaum), on design and
%analysis of galaxy redshift surveys (Ho, Padmanabhan, Samushia, Wang, Weinberg),
%on space-based slitless spectroscopy analogous to that planned for WFIRST
%(Teplitz), on photometric calibration (Padmanabhan), on photometric
%redshifts (Capak) from large imaging surveys, and on cosmological forecasting
%%and parameter estimation from combinations of cosmic microwave
%background (CMB), WL, and LSS data (Bean, Dor\'e, Eifler,
%Hirata, Ho, Jain, Mandelbaum, Samushia, Spergel, Wang, Weinberg).

%Through this team of Co-Is, we have close connections to most of the
%major current or planned cosmological experiments that will provide
% This team of Co-Is brings close connections to most of the
%major current or planned cosmological experiments that will provide
%the context for the WFIRST dark energy program. This includes the WMAP and Planck CMB missions,
%the Sloan Digital Sky Survey (SDSS), the Baryon Oscillation Spectroscopic Survey (BOSS), the Dark Energy Survey (DES),
%the Subaru Hyper Suprime-Cam (HSC) and Prime Focus Spectrograph (PFS)
%projects, the Dark Energy Spectroscopic Instrument (DESI), the
%Euclid mission and the Large Synoptic Survey Telescope (LSST) Dark
%Energy Science Collaboration (DESC). Our team of U.S. and
%international collaborators brings extensive expertise in detector
%characterization, cosmological simulations, detailed simulations of
%observational data sets, and the theoretical modeling and cosmological
%interpretation of weak lensing and galaxy clustering data.
%Notably, members of our team are responsible for nearly all of the tables
%and figures in \S\S\ 2.2.3-2.2.5 of the SDT15 report, describing the
%HLS dark energy program.  We therefore have an unparalleled understanding
%of the current design of WFIRST-AFTA and of the
%challenges ahead in achieving its science goals.
