 %1-2 paragraph overview of science goals.  BAO, RSD, other applications.
 %SDT2015 as starting point.

As discussed extensively in \S 2.2.4 of SDT15 (written by
members of our team), the defining goal of HLS spectroscopy is to derive
constraints on dark energy from a slitless spectroscopic (grism)
redshift survey of approximately 20 million emission line galaxies (ELG) in the redshift range $z=1-3$.
The galaxy redshift survey will enable high-precision measurements of the cosmic expansion history via BAO and structure growth via RSD.  
Acoustic oscillations in the pre-recombination universe imprint a characteristic scale on matter clustering, which
can be measured in the transverse and line-of-sight directions to
determine the angular-diameter distance $D_A(z)$ and Hubble parameter $H(z)$, respectively \cite{Blake03,Seo03,CW12}.  Anisotropy of clustering
caused by galaxy peculiar velocities constrains (in linear perturbation
theory) the combination $\sigma_m(z) f_g(z)$, where $\sigma_m$ describes
the rms amplitude of matter fluctuations and $f_g(z) \equiv d\ln\sigma_m(z)/d\ln a$ is the fluctuation growth rate.
Thus the GRS on its own can address the key questions identified by
NWNH: whether cosmic acceleration is caused by modified gravity
or by dark energy, and whether (in the latter case) the dark energy
density evolves in time \cite{Guzzo08,Wang08}.  These tests become more powerful in
combination with weak lensing and cluster measurements from HLS Imaging
and high-precision relative distance measurements from the Supernova
Survey \cite{dePutter:2013xda,dePutter:2013nha}. The broadband shape of the galaxy power spectrum and higher order
measures of galaxy clustering provide additional diagnostics of
dark energy, neutrino masses, and inflation, and insights on the physics of galaxy formation.  
%There are two largely distinct sources of systematics in the galaxy clustering program, associated with
%the uniformity of the GRS and with astrophysical modeling uncertainties.
%We discuss these in \S\ref{sec:grs_requirements} and~\S\ref{sec:grs_forecasting}, respectively, and we briefly
%discuss mitigation strategies in \S\ref{sec:grs_mitigation}.
While all aspects of our GRS investigation are interconnected, we
organize it in a structure similar to that of \S
\ref{sec:wl_gal-clusters} for clarity: requirements, simulations, and prototype pipelines in \S \ref{sec:grs_requirements},
cosmological forecasting, modeling, and cosmological simulations in \S
\ref{sec:cmethods}, and systematics testing and mitigation in \S \ref{sec:gal_syst}.


\subsection{Requirements}
\label{sec:grs_requirements}
\Auth{Yun, Lado}

The most important task of the SIT in guiding
development of the WFIRST HLS spectroscopy is to set and validate the
requirements of the instrument, the data reduction software, and the survey.
The GRS Lead Co-I Wang will work closely with PI Dor\'e and Co-Is
Hirata and Teplitz on setting requirements for the GRS. To make fully
informed decisions, the team requires high fidelity simulations of
both
instrument performance and the observable sky that the instrument will
measure.  The team must also ensure that the analysis of these data by
the reduction pipeline will be of sufficient quality to enable
measurement with the high precision needed for cosmology.
These simulation and pipeline activities will require the team to
coordinate with the WSC.  Several members of our team (Wang, Teplitz,
Capak, Helou) are located at the Infrared Processing and Analysis
Center (IPAC), and work closely with the WSC.
%\Oli{Mention George too if he stays in}%The WSC task plans are still under discussion, but will
%most likely include development of a pixel-level simulator for imaging
%and spectroscopy.  They will (certainly) include the construction of a
%production-quality analysis pipeline to remove detector effects and
%(probably) extract and measure emission-line spectra.
To the extent practical, we will draw on tools created by the WSC and
design our own software and simulations to be useful to them.
We note that our work primarily demands the ability to quickly and flexibly
simulate different configurations and analyze the results with different
algorithms, while the WSC has the task of developing tools for the
community and production
ready pipelines that integrate with the full WFIRST data system.

\subs{Deriving requirements.} As with WL (\S\ref{sec:wl_requirements}), we will focus first on GRS
requirements that may drive hardware choices, i.e., those that may be
demanding in terms of grism design, detector properties, stability and repeatability of pointing, or dedicated calibration hardware.
We will include a prioritized list of effects to
incorporate in grism simulations.  Over time, we will use our
increasingly realistic network of simulations to evaluate the impact
of requirements and possible trades from the pixel level through
to cosmological inferences. The starting points for this
process are the WFIRST ETC and survey planning software
and the \CoLi\ forecasting tool.

The maximum achievable statistical power of the GRS is determined
mainly by the telescope aperture, throughput, detector area and pixel
size, and allotted observing time.  However, the statistical power and
uniformity of the GRS are further affected by numerous aspects of
instrument performance and survey design, e.g., spectral resolution,
detector read noise and persistence, dither and roll angle pattern,
image quality, complexity of non-1$^{\rm st}$ order features,
spatially varying thermal background due to the warm telescope,
scattered light from bright stars, repeatability of the grism
positioning, and accuracy of calibration of the wavelength-dependent
PSF and distortion map.

Non-uniformity of the survey, which is inevitable to some degree, can
be corrected in clustering measurements by weighting galaxies to
account for incompleteness.  However, large corrections typically come
at a cost in statistical power, and imperfect knowledge of the
non-uniformity leads to systematic errors in the inferred clustering.
The other important source of observational systematics is
contamination of the redshift catalog by artifacts or objects with
incorrectly determined redshifts, and loss of objects from the catalog
because of catastrophic redshift errors or uncertainties in the flux
calibration.  We will define requirements such that (a) the
statistical power of the GRS is close to the maximum allowed by the
telescope aperture and detector area and (b) the impact of uncorrected
observational systematics is small compared to the statistical
errors. The expected precision of the galaxy power spectrum provides a
useful guide to the statistical power of the GRS, but the full
question of cosmological constraining power depends on the
astrophysical modeling techniques used to interpret the measured
clustering, as discussed in \S\ref{sec:cmethods} below.  By the end of our
investigation we will have a complete set of tools to evaluate the
impact of hardware or strategy trades, changes in requirements, or
changes in astrophysical inputs on the expected cosmological return
from the GRS.

\subs{Simulations.} Our development of requirements and a prototype spectroscopic
pipeline will rely critically on realistic simulations of the
pixel-level grism images.  We will work closely with the WSC on
developing these simulations for a variety of cases, ranging from
simple widely separated sources to realistically clustered galaxy
populations drawn from the cosmological simulations described in
\S\ref{sec:cmethods}.  Our team has extensive experience in producing
such simulations for the Hubble Space Telescope (HST) and Euclid.  Co-I Tepliz is one of the leaders
of the HST Wide-Field Camera 3 (WFC3) IR Spectroscopic Parallel survey (WISPs), for which
pixel simulations are vital in assessing completeness and other
parameters \cite{Colbert13}; Co-I Wang (with Teplitz and Capak) is
developing simulation techniques for the Euclid grism survey.
Co-I Teplitz will lead our grism simulations for WFIRST.
A critical astrophysical input for these simulations is the
redshift-dependent luminosity function of H$\alpha$ and [OIII]
emitters, which is currently uncertain at levels that have an
important impact on WFIRST strategy and performance forecasts.  Co-Is
Teplitz and Wang are part of an HST archival study to reprocess
existing data from multiple HST projects to mitigate systematic
uncertainties of the H$\alpha$ luminosity function (LF) measurement.  Through WISPs, Teplitz
is also working to obtain significantly more HST data to improve the
H$\alpha$ LF measurement.  In addition, realistic galaxy templates are
vital to the forecasting of grism measurements, and we are working to
improve both line diagnostics and prediction of line vs. continuum
properties.

\subs{Prototype pipeline.} We will build a prototype spectroscopic pipeline for the analysis of
slitless spectroscopic data to produce a redshift catalog.  This is a
complex, multi-step process.  Co-I Teplitz has extensive experience
with HST slitless spectroscopy using the WFC3, HST Near Infrared Camera and
Multi-Object Spectrometer (NICMOS), and HST Space Telescope Imaging
Spectrograph (STIS) instruments \cite{Atek10,Shim09,Teplitz03}; he
will lead our work on prototype pipelines for the GRS. We will take the basic steps
implemented for WFC3 processing as the starting point for a prototype
WFIRST pipeline. This pipeline must clean the grism images of contamination
from detector artifacts and cosmic rays, register and combine images
from separate dithers and roll angles, match objects in the dispersed
and direct imaging exposures, extract wavelength- and flux-calibrated
2D spectra, infer redshifts based on detected emission lines, and
measure emission-line fluxes and other spectroscopic characteristics.
The resulting catalogs are the input for the clustering analyses
discussed further in \S\ref{sec:cmethods}.

\subs{Analysis challenges.} Slitless spectroscopic analysis presents several important challenges.
First, the pipeline must mitigate the confusion caused by overlapping
spectra.  The standard solution (used by the HST data pipeline) is to
subtract a model of neighboring objects from each source as it is
extracted.  We will investigate the use of HST-like algorithms for
WFIRST, as well as more sophisticated solutions that could produce
better results, such as fitting the full pixel set for regions of the
frame.  Model dependent solutions, with iterative fitting, are
potentially promising, but biases would have to be carefully
understood.  While the HLS obtains exposures at multiple roll angles,
these may be greatly separated in time.  This could introduce new
problems for variable sources, or in fields with foreground moving
objects.  We will also develop methods to automate quality assessment
and flagging of extracted spectra, as the sheer volume of WFIRST GRS
will make human review of spectra (standard practice in current grism
surveys) impossible.

A second major challenge is the need to mitigate catastrophic redshift
errors.  Such failures arise from the misidentification of redshifts
(e.g., confusing [OIII] for H$\alpha$\ in low signal to noise (S/N) spectra) or
false-positive line detections caused by noise peaks or unflagged
cosmic rays.  Redshift fidelity can be greatly improved by using the
photometric redshift estimates derived from the multi-band photometry as a prior
in the redshift determination.  Co-I Capak is spearheading
multidimensional analysis of galaxy color information for WFIRST
photometric redshifts, and that work will be folded into the
spectroscopic pipeline.
% (\Oli{PETER SHOULD DOUBLE CHECK THE WORDING HERE}).

\subs{Completeness maps.} Nearly as important as the redshift catalogs themselves is the production of
completeness maps that characterize the spatially varying depth of the survey,
the level of contamination, and regions that should be masked because the
catalog is unreliable.  These completeness maps are used to weight galaxies in
clustering analysis and/or to create random catalogs such that the local number
density of points is proportional to the likelihood of successfully measuring a
redshift of a galaxy if it were at that point.  Co-I Samushia is
leading a similar effort in DESI and has previously worked on quantifying and removing
systematic effects associated with inaccuracies in random catalogues
\cite{Samushia2012}.
Co-I Ho has also led the effort in creating the SDSS-BOSS LSS catalog and randoms \cite{Reid2015} and led the effort in removing observational effects in BOSS LSS catalog \cite{Ross2011}.
Co-I Samushia will lead our work to develop tools for creating these completeness maps
by a full forward-modeling method, where artificial sources are assigned random
angular positions and redshifts, added to grism images, and pushed through the
data pipeline.  Compared to existing large redshift surveys (from ground-based
fiber spectroscopy), the WFIRST completeness map will have much more complex
small scale structure because of the varying numbers of exposures at individual
points on the sky and sensitivity variations across the focal plane.  Because of
source confusion and sky background effects, the completeness and contamination
will be a function of the local galaxy surface density.  We will develop
strategies and tools for recording these large and complex completeness maps in
formats that can be efficiently used to create random catalogs and weight
galaxies for clustering analyses.  By the end of the investigation period, we
will be able to create full pixel-level simulations from an input cosmological
simulation (see \S\ref{sec:cmethods}), run them through our proto-type pipeline to create a
redshift catalog and completeness map, and analyze the resulting artificial data
set with our clustering analysis tools to compare to the idealized case that has
the complete galaxy catalog of the cosmological simulation.

\subs{Calibration strategies.}
We will define the absolute and relative calibration requirements for the GRS, such as the angular scale and temporal stability.
We will develop methods for calibrating the relative and absolute flux measurements along with wavelength calibration and redshift accuracy and completeness.
For flux and wavelength calibration we will set requirements on the ground testing, in flight calibration sources, and calibration observations based on experience
with other missions including Spitzer, HST, and Euclid.  Furthermore, we will investigate self calibration strategies based on optimizing dither patterns and exposure
times for the science observations and the use of touch-stone fields that both calibrate and provide long-term trending of the data.  Both the primary and
self calibration procedures will be tested with simulations specified by this SIT and conducted by the WSC.  Finally, we will use the large spectroscopic surveys
necessary for the weak lensing photo-$z$ calibration to verify the calibration by directly testing the redshift accuracy and completeness estimates from the simulations.
Co-Is Capak and
Padmanabhan, both with extensive experience from similar work for
Euclid and BOSS, will lead our calibration work.

\subsection{Simulations}
\label{sec:grs_simulations}
\Auth{Shirley, Elena, Andrew, Alina, Alex, Yun}


%===
% \subsection{Requirements, Simulations, and Proto-type Pipelines}


% The most important task of the SIT in guiding
% development of the WFIRST HLS spectroscopy is to set and validate the
% requirements of the instrument, the data reduction software, and the survey. To make fully informed
% decisions, the team requires high fidelity simulations of both
% instrument performance and the observable sky that the instrument will
% measure.  The team must also assure that the analysis of these data by
% the reduction pipeline will be of sufficient quality to enable
% measurement with the high precision needed for cosmology.
% These simulation and pipeline activities will require the team to work
% closely with the WFIRST Science Centers (WSC).  Several members of our team
% (Wang, Teplitz, Capak) are located at IPAC, and they work closely with
% the WSC. %The WSC task plans are still under discussion, but will
% %most likely include developmet of a pixel-level simulator for imaging
% %and spectroscopy.  They will (certainly) include the construction of a
% %production-quality analysis pipeline to remove detector effects and
% %(probably) extract and measure emission-line spectra.
% To the extent practical, we will draw on tools created by the WSC and
% design our own software and simulations to be useful to the WSC.
% We note that our work primarily demands the ability to quickly and flexibly
% simulate different configurations and analyze the results with different
% algorithms, while the WSC has the task of developing production
% ready pipelines that integrate with the full WFIRST data system.

% As with WL (\S\ref{sec:wl_requirements}), we will focus first on GRS requirements that may drive hardware choices, i.e., those that may
% be demanding in terms of grism design, detector properties, stability
% and repeatability of pointing, or dedicated calibration hardware.
% Our initial assay will include a prioritized list of effects to incorporate in grism simulations.  Over time, we will use our
% increasingly realistic network of simulations to evaluate the
% impact of all requirements and possible trades from the pixel level through to cosmological inferences.  The starting points for this
% investigation are the WFIRST ETC and survey planning software (both written by Co-I Hirata) and the CosmoLike forecasting tool
% described in \S\ref{sec:wl_requirements}.

% The maximum achievable statistical power of the GRS is determined
% mainly by the telescope aperture, throughput, detector area and pixel size,
% and allotted observing time.  However, the statistical power and uniformity of the GRS
% are further affected by numerous aspects of instrument performance and survey design:
% spectral resolution, detector read noise and persistence, dither and roll angle pattern,
% blurring due to diffraction by the glass elements of the grism, spatially varying thermal
% background due to the warm telescope, scattering due to bright
% stars, repeatability of the grism positioning, and accuracy of
% calibration of the wavelength-dependent PSF and distortion map.

% Non-uniformity of the survey, which is inevitable to some degree,
% can be corrected in clustering measurements by weighting galaxies
% to account for incompleteness.  However, large corrections typically come
% at a cost in statistical power, and imperfect knowledge of the
% non-uniformity leads to systematic errors in the inferred clustering.
% The other important source of observational systematics is
% contamination of the redshift catalog by artifacts or objects with incorrectly
% determined redshifts, and loss of objects from the catalog
% because of catastrophic redshift errors.
% We will define requirements such that (a) the statistical power of
% the GRS is close to the maximum allowed by the telescope aperture
% and detector area and (b) the impact of uncorrected observational
% systematics is small compared to the statistical errors. The expected precision of
% the galaxy power spectrum provides a useful guide to the statistical power of the GRS, but the full
% question of cosmological constraining power depends on the
% astrophysical modeling techniques used to interpret the measured
% clustering, as discussed in \S 5.2 below.
% By the end of our investigation we will have a complete set of
% tools to evaluate the impact of hardware or strategy trades,
% changes in requirements, or changes in astrophysical inputs
% on the expected cosmological return from the GRS.

% Our development of requirements and a proto-type spectroscopic pipeline
% will rely critically on realistic simulations of the pixel-level
% grism images.  We will work closely with the WSC on developing
% these simulations for a variety of cases, ranging from simple
% widely separated sources to realistically clustered galaxy populations
% drawn from the cosmological simulations described in \S 5.2.
% {\bf Can we say more here?  Do we have any experience producing
% grism simulations for HST analysis?}
% A critical astrophysical input for these simulations
% is the redshift-dependent luminosity function of H$\alpha$ and [OIII]
% emitters, which is currently uncertain at levels that have
% an important impact on WFIRST strategy and performance forecasts.
% Co-I's Teplitz and Wang are part of an HST archival study to reprocess existing
% data from multiple HST projects to mitigate systematics uncertainties of the
% H$\alpha$ LF measurement.  Teplitz is also a key member of the WISP team,
% which aims to obtain significantly more HST data to improve the H$\alpha$ LF measurement.

% We will build a prototype spectroscopic pipeline for the
% analysis of slitless spectroscopic data to produce a redshift catalog.
% This is a complex, multi-step process.
% Co-I Teplitz has extensive experience with the HST grism data
% pipeline (e.g., \cite{REFS}), and we will take the algorithms
% implemented in this pipeline as the starting point for a
% proto-type WFIRST pipeline. This pipeline must clean the grism images for contamination from
% detector artifacts and cosmic rays, register and combine images from
% separate dithers and roll angles, extract wavelength- and flux-calibrated
% 2D spectra, match these spectra
% to individual objects from the photometric imaging data,
% infer redshifts based on detected emission lines, and measure
% emission-line fluxes and other spectroscopic characteristics.
% The resulting catalogs are the input for the clustering analyses
% discussed further in \S 5.2.

% The biggest challenge for slitless spectroscopic analysis is
% eliminating confusion from overlapping spectra.
% The standard solution (used by the HST data pipeline) is to subtract a
% model of neighboring objects from each source as it is extracted.
% We will investigate the use of HST-like algorithms for WFIRST,
% as well as more sophisticated solutions that could produce better results,
% such as fitting the full pixel set for regions of the frame.
% Model dependent solutions, with iterative fitting, are potentially
% promising, but biases would have to be carefully understood.
% While the HLS obtains exposures at multiple roll angles, these may
% be greatly separated in time.  This could introduce new problems for variable sources, or in fields with foreground moving objects.
% We will also develop methods to automate quality assessment
% and flagging of extracted spectra, as the sheer volume of WFIRST GRS will make human review of spectra
% (standard practice in current grism surveys) impossible.

% Another challenge is the misidentification of noise peaks as emission lines, which leads to
% catastrophic redshift errors. This can be mitigated by using the photometric redshift derived from
% the multi-band photometry as a prior in the redshift determination.

% Nearly as important as the redshift catalogs themselves is the
% production of completeness maps that characterize the spatially
% varying depth of the survey, the level of contamination, and
% regions that should be masked because the catalog is unreliable.
% These completeness maps are used to weight galaxies in clustering
% analysis and/or to create random catalogs such that the local number
% density of points is proportional to the likelihood of successfully measuring a redshift of a galaxy if it were at
% that point.  We will develop tools for creating these completeness
% maps by a full forward-modeling method, where artificial sources
% are assigned random angular positions and redshifts, added to
% grism images, and pushed through the data pipeline.
% Compared to existing large redshift surveys (from ground-based
% fiber spectroscopy), the WFIRST completeness map will have much
% more complex small scale structure because of the varying numbers
% of exposures at individual points on the sky and sensitivity
% variations across the focal plane.
% Because of source confusion and sky background effects,
% the completeness and contamination will be a function of
% the local galaxy surface density.
% We will develop strategies and tools for recording these large and
% complex completeness maps in formats that can be efficiently
% used to create random catalogs and weight galaxies for clustering
% analyses.  By the end of the investigation period, we will be able
% to create full pixel-level simulations from an input cosmological
% simulation (see \S 5.2), run them through
% our proto-type pipeline to create a redshift catalog and completeness
% map, and analyze the resulting artificial data set with our
% clustering analysis tools to compare to the idealized case
% that has the complete galaxy catalog of the cosmological simulation.


\subsection{Cosmological Forecasting, Modeling, and Simulations (D2, D8, D9)}
\label{sec:cmethods}
%================================================
%\Auth{Yun, Katrin, Nikhil, Rachel B, Alina, Shirley, Olivier}

\subs{Forecasting.} Our initial forecasts for the cosmological constraints from 
the WFIRST GRS will adopt the model-independent approach (incorporating
both BAO and RSD) that Co-I Wang has developed \cite{Wang2013} and
used for the WFIRST SDT reports and similar forecasts for Euclid \cite{WangEuclid2010}.
This approach offers a fast way to forecast how uncertainties in
$H(z)$, $D_A(z)$, growth rates, and other cosmological parameters
change in response to changes of the survey strategy, instrument
performance, or astrophysical inputs such as the H$\alpha$ luminosity
function.  Early in this investigation, we will incorporate a full description
of redshift-space galaxy clustering into \CoLi, using a halo
occupation density (HOD) framework similar to that already implemented for 
angular galaxy clustering \cite{Krause2013}.  In the medium term, we will also
incorporate effects of clustering measurement and theoretical
modeling systematics via nuisance parameters, analogous to our
existing treatments of observational and theoretical systematics
in weak lensing analysis.  The expected level of these systematics
will be informed by the studies described in \S\ref{sec:grs_requirements}
and below.  As with the imaging survey, this comprehensive forecasting
framework will enable us to connect low-level technical requirements
to our top-level science goals.

\subs{GRS modeling.} The development of the methodology for the
interpretation of GRS data is centered on the mitigation of the
astrophysical systematic effects  for galaxy clustering measurements:
nonlinear effects, RSD (growth rate signal on large scales and
contamination on small and intermediate scales),  and galaxy bias (the
difference between galaxy and matter distributions). 
Co-I Padmanabhan is a leading expert in BAO/RSD data analysis \cite{Padmanabhan2007,Padmanabhan2008,Padmanabhan2009,Padmanabhan2012,Xu2012}; he will lead our
work in GRS modeling/interpretation methods, with participation from the PI and Co-Is
Bean, Ho, Samushia, Spergel, Wang, and Weinberg \cite{CW12,Ho2012,Anderson2014,Wang2014,Osumi2015,Alam2015a,Alam2015b,Cuesta2015}.

BAO measurement is now a mature field, but WFIRST probes new regimes
of precision and redshift using different instrumental choices
and different classes of galaxy tracers from previous surveys.
Effects of non-linear clustering and galaxy bias are expected to
influence BAO measurements at the $\sim 0.5\%$ level \cite{Padmanabhan2009}, which is
significant compared to WFIRST statistical errors.
Reconstruction methods \cite{Eisenstein2007,Padmanabhan2012,Vargas2015}, 
which attempt to reverse the
nonlinear evolution of the BAO feature, appear to remove
most of this effect while simultaneously improving the precision
of BAO measurements.  Current observational studies use very simple
reconstruction algorithms.  We will explore more sophisticated
reconstruction methods, building on low redshift work \cite{Carrick15}, and test their performance on simulations
of WFIRST galaxy redshift catalogs, including realistic treatments of
survey geometry, redshift evolution, source space density, and
variable completeness.  Building on current work by Co-Is
\cite{Padmanabhan2012,Vargas2015,Osumi2015,Zhu2015}, 
we will also investigate improved clustering estimators that can sharpen the precision and improve the robustness
of BAO measurements.

In sharp contrast to BAO measurements, cosmological inference from 
RSD measurements is already limited mainly by uncertainties in 
theoretical modeling, with application of different models to the
same underlying data yielding differences at the $\sim 10\%$ level.
%{\bf SH: Not sure it is 5\% level, I think BOSS results differ by 
%nearly 10\% level with the mock challenge} .
Furthermore, the statistical signal-to-noise ratio of RSD measurements
increases rapidly with decreasing scale, so there are potentially
large gains from modeling that extends into the fully non-linear regime.
We will pursue a variety of approaches to improving and testing
RSD models, including the 
%{\bf not calling this brute force ? } ``brute force'' 
efficient method of computing predictions
numerically by populating the halos of N-body simulations with galaxies. 
Co-I Spergel has expertise in combining imaging and spectroscopic 
data to predict the relationship between galaxies and halos \cite{Hikage2012}.
One can think of this method as producing ``emulators'' \cite{Heitmann2014}
that predict galaxy clustering statistics as a function 
of cosmological parameters and parameters that describe the relation
between galaxies and dark matter halos.
The approach shows promise (e.g., \cite{Reid2014}),
but it relies on parameterized models for populating halos,
and the accuracy of these needs to be tested against galaxy catalogs
constructed in ways that do not share the same assumptions
(e.g., by semi-analytic models or abundance-age matching).
We will use similar techniques to investigate the impact of
non-linear evolution and bias on the broadband galaxy power
spectrum, and thus improve our ability to extract cosmological
information from this measurement.

Galaxy bias is likely scale-dependent; its testing will require realistic ELG mocks, and its mitigation will require 
the successful measurement of the higher-order statistics of galaxy clustering, which in turn requires a 
sufficiently high galaxy number density for the GRS. 
An essential difference between the WFIRST and Euclid spectroscopic
surveys is that due to the much smaller pixel scale (0.11$^{\prime\prime}$ for WFIRST
versus 0.3$^{\prime\prime}$ for Euclid) and larger telescope aperture, WFIRST 
is capable of carrying out a significantly deeper GRS, which can result in
%, at least as currently planned, is the 
a much higher space density of the WFIRST sample over most
of its redshift range.
This high sampling density represents a significant science opportunity
for WFIRST; 
in particular it will boost the significance of higher-order correlations. 
Building on previous work by members of our team \cite{Takada08,Schaan14,Dore14,Chen15},
%(Takada et al. 2008; Schaan, Takada \& Spergel 2014; Chen, Ho et al. 2015), 
we will investigate techniques that use the galaxy
bispectrum and other higher-order statistics to sharpen cosmological
constraints, by directly probing the matter density and velocity fields
and by improving knowledge of ``nuisance parameters'' that describe
galaxy bias.  We will examine ways that RSD measurement precision
can be improved by cross-correlating multiple tracer populations
with different clustering bias to suppress cosmic 
variance \cite{mcDonald2009,Bernstein2011,dePutter:2014lna}, weighting galaxies by mass
to suppress shot noise \cite{Seljak2009}, and building group catalogs
to collapse fingers-of-God \cite{Reid2010}.
We will investigate potential gains from cross-correlating the
WFIRST galaxy redshift catalogs with H$\,${\sc i} 21 cm ``intensity mapping''
measurements, or with CMB measurements, or (at $z>2$) the Ly$\alpha$ forest.
In all of these studies, we will pay particular attention to 
the influence of the sampling density, as this directly informs
the trade between depth and area in the GRS (see \S\ref{sec:sur_opt}).

\subs{Cosmological simulations.} The cosmological simulations described in \S\ref{sec:wl_methodology} 
will also be useful for the methodology development outlined above.
However, the optimal simulations for BAO and RSD studies will typically
be larger volume and lower resolution than those for weak lensing:
large volumes are needed for good statistics and to eliminate finite
box effects, but we do not need to model the small scale matter 
distribution or baryonic effects (which are encoded in the 
models used to populate halos with galaxies).  Furthermore, 
simulations tuned to the WFIRST GRS need only be evolved to $z=1$.
Co-I Ho will lead our cosmological simulations for the GRS, with participation from
team members Benson, Heitmann, Kiessling, Wang, and postdocs.

We will leverage the participation of several of our team members in 
Euclid and DESI to produce large simulated ELG catalogs,
building on work we have already begun for these projects.
To model emission-line selection, we will use both semi-analytic galaxy
formation models (SAM) and HOD models that are tuned to produce
observed number densities and clustering.
%The current Euclid ELG mocks are made using SAM, but are limited in emission line modeling 
%and simulation volume. 
We expect to be able to provide ELG mocks for WFIRST similar to those used 
by Euclid on a short time scale.
We will incorporate these into the early pixel-level 
simulations described in \S\ref{sec:grs_requirements},
which will in turn be used to assess impacts of incompleteness,
contamination, and redshift errors on the galaxy distribution.

We will base our first cosmological volume ELG catalogs on two very large simulations
that are already available to us through collaborator Heitmann: 
the Outer Rim simulation, covering
a volume of $4.225\,{\rm Gpc}^3$ with a particle mass of
$\sim 2\times 10^9 M_\odot$,
and the Q Continuum simulation, covering a volume of $1.3\,{\rm Gpc}^3$
with particle mass of $\sim 10^8 M_\odot$.
The Outer Rim simulation was used to create the simulated ELG
catalog for DESI.
We will combine the halo populations from these simulations with the
Galacticus SAM code \cite{Benson2010} to create clustered ELG catalogs,
some in fixed-redshift cubes for methodology tests and some in light cones
with realistic survey geometry and redshift evolution.
Our most ambitious simulation efforts, later in the investigation
period, will take full light-cone ELG catalogs, create pixel-level
simulations that span the entire HLS area, and analyze these simulations
with the proto-type pipeline to produce ``observed'' galaxy catalogs.
We can then apply the full clustering measurement machinery 
(including corrections for varying completeness) to these catalogs
and apply our cosmological inference tools 
to understand the impact of observational systematics on
cosmology from the WFIRST GRS.
As with the weak lensing investigation, we will investigate computational
requirements beyond FY20 and techniques to reduce them. We will also make our
simulations publicly available so that others can develop and
test their own methods, with some of them released in the form of blind
data challenges.

\subsection{Systematics Testing and Mitigation (D8)}
%================================================
%\Auth{Shirley, Olivier}
\label{sec:gal_syst}
%\subsection{Systematics Testing and Mitigation}

%{\it Rewriting this entirely differently from before according to DW's suggestions: mostly observational systematics, do not currently touch theoretical systematics }

The galaxy clustering measurements are susceptible to observational and astrophysical systematic effects. 
We discussed the astrophysical systematic effects and their mitigation in \S\ref{sec:cmethods}.
We now focus on the observational systematic effects.

As we discussed in the \S\ref{sec:cmethods}, we will take full light cone ELG catalogs,
create pixel-level simulations that span the entire HLS area, and produce the
observed galaxy and corresponding random catalogs.  We envision that multiple
LSS catalogs will be produced by extracting the galaxies (and their
corresponding randoms) according to their specific continuum levels
and/or line-fluxes (or other selection criteria). For each of these catalogs, we will
test for systematics such as effects of stellar density, dependencies on line
luminosity, continuum luminosity and varying exposure number. Co-I Ho led
investigations in effects of observational  systematics on galaxy over-density in
BOSS \cite{Ho2012}; we will apply the same techniques in detecting galaxy
over-density variations due to the various potential systematic sources.  
Co-I Ho will lead our work to design and perform internal empirical tests that involve dividing the
galaxies (and corresponding randoms) into subsets of different continuum
luminosity, line luminosity, galaxy environment, exposure number and other
relevant parameters. We will pay particular attention to potential systematic
effects caused by the complex  structure of the completeness function, as a
function of redshift. This is particularly important as the number of exposures
can vary significantly, from 0 to 10 with a median of 7 in the SDT15
observing strategy.

Team Co-Is have experience in using template projection method
and cross-correlation method in photometric clustering to mitigate observational
systematics such as stellar density, PSF variations, magnitude error
fluctuations \cite{Pullen2013,Agarwal2014}.  We will
adapt these methodologies to remove systematics in 3D clustering. We will 
test our 3D systematics detection and mitigation methodology on our
simulated catalogs and check whether we achieve unbiased results
in BAO distances and the growth rate of large scale
structure.  These potential systematics will also affect the photometric
clustering and the full shape of the galaxy power spectrum,
which can be powerful in constraining
the sum of neutrino masses. 

%{\it BAO systematics}: 
%The robustness and accuracy of the BAO method derive from the 
%simplicity of the early Universe and the precision with which we know the speed and 
%time of propagation of sound waves in the primordial plasma. The evolution of density fluctuations in the Universe
%is very well described by linear perturbation theory  and is now exquisitely tested 
%by the recent measurements of temperature fluctuations in the Cosmic Microwave
%Background radiation by the {\it Planck} satellite. The current CMB measurements
%constrain the size of the BAO standard ruler to much better than $0.5\%$. Furthermore, 
%any mis-calibrations in the acoustic scale would affect principally the determination 
%of the Hubble constant, not the dark energy constraints \citep{2004PhRvD..70j3523E}.
%
%The sound waves travel a comoving distance of 150 Mpc, setting the BAO scale to be much
%larger than the scale of gravitational collapse even 
%in the present Universe (about 10 Mpc).
%Analytical calculations, verified by direct numerical simulations, have found the nonlinear evolution of the
%density field alters the BAO scale by less than $0.5\%$ at the present epoch (REF by Nikhil, Martin) , and
%even less at the higher redshifts probed by WFIRST. 
%Galaxy formation may also result in an additional shift in the BAO scale 
%due to mismatched weighting of high and low density regions. 
%Initial perturbative
%and numerical work also find these shifts to be small, with the
%most extreme shifts less than $0.5\%$ (Eisenstein 08, Tojeiro 14, Ross 15). 
%
%{\bf SH: NEAR TERM plan needed  }
%
%Modeling based just on current theory and simulations could in principle reduce them below 0.1\%. 
%Furthermore, as we
%discuss below, (partial) reconstruction of the initial density field may
%reduce these effects below the 0.1\% level without the need for further
%modeling (Eisenstein, Seo, Spergel, Sirko 08, Padmanabhan et al. 12, Vargas-Magana, Ho 14) . In addition, the WFIRST target samples are designed to overlap in multiple redshift
%ranges, allowing empirical tests of the robustness of the BAO measurements to
%different tracer populations.
%All of the above strongly argue that the theoretical systematic effects associated 
%with the BAO measurements are either intrinsically or correctable to below the $0.1\%$ level. 
%
%As an example, BOSS has found all of the theoretical systematics in BAO to be each at $\approx 0.1\%$ or less (Vargas-Magana, Ho et al. 2014), while the observational systematics are kept at below $\approx 0.5\%$ with corrections using weights for each of the galaxies (Ross, Ho et al. 2012, Ross et al. 2014). 
%
%{\it RSD systematics}: 
%Galaxies are expected to follow the same gravitational
%potential as the dark matter and hence have the same velocities. 
%The effects from the gravitational potential is detectable in redshift surveys, because the redshift of the galaxy provides information not only on the radial distance, but also on the radial velocity through the Doppler shift.
%This induces anisotropies in the clustering, which are generically called redshift space distortions (RSD). 
%They provide an opportunity to extract information on the dark matter clustering directly.
%On large scales clustering of galaxies along the line of sight is enhanced
%relative to the transverse direction due to peculiar motions and this allows
%one to determine the ratio of logarithmic rate of growth $f$ to bias $b$.  Combining
%the statistics from different lines of sight
%one can eliminate the unknown bias and measure
%directly the logarithmic rate of growth times the amplitude.
%
%The main theoretical systematic uncertainty in RSD is that nonlinear velocity effects extend to rather large scales and
%give rise to a scale-dependent and angle-dependent clustering signal. It is easy to see these effects in any
%real redshift survey: one sees elongated features along the line of sight,
%called the fingers-of-god (FoG) effect,
%which are caused by random velocities inside virialized objects such as
%clusters, which scatter galaxies
%along the radial direction in redshiftsift space, even if they have a localized
%spatial position in real space.
%This is just an extreme example and other related effects, such as nonlinear infall streaming
%motions, also cause nonlinear corrections. In addition, RSD measure velocities as sampled at the 
%galaxy positions. One is thus probing not the velocity field, but rather the momentum density field. 
%Galaxies are a biased tracer of the dark matter and this introduces scale dependent effects into 
%RSD statistics even if galaxies are simply a linear tracer of the dark matter. 
%Nonlinearities in the density and velocity fields, as well as galaxy biasing, can induce 10\% effects on RSD at $k \sim 0.1$~$h$/Mpc.  Current models of RSD are able to reproduce these nonlinear effects at the percent level for $k<$~0.05--0.1~$h$/Mpc.  Extending this to smaller scales would increase the power of the RSD component of WFIRST. This will require us to improve our bias models and the realism of our simulations.
%
%{\bf SH NEAR TERM task needed} 
%Most of the observational systematics examined in detail in the SDSS-III BOSS 
%\cite[see][]{Ross12} primarily affect clustering on the largest scales;
%currently these are of little concern for RSD measurements, for which the signal
%comes primarily from the smallest scales included in the measurements.  
%Multiple teams within BOSS has checked the influence of observational systematics on RSD and found it to be at the level of $0.5\sigma$ (Samushia et al. 2014, Alam, Ho et al. 2015).
%One of the most
%important systematic effects is the estimate of a survey's radial selection
%function \cite{SamPerRac11,Ross, Ho et al.  12}.  Since the redshift distribution of targets
%cannot be predicted precisely a priori, it must be measured directly from the
%observed galaxies' redshift distribution.  Doing so removes some cosmological
%radial modes from the observed galaxy over density field, resulting in a bias in
%the monopole-quadrupole amplitudes at the $<0.2 \sigma$ level.  The ratio of
%systematic to statistical uncertainty should remain relatively constant with
%survey area for a given redshift distribution, since the statistical errors on
%the correlation function and $n(z)$ shrink at the same rate.
%
%


%{\it Photometric Clustering Systematics} 
%


%\bi
%\item Describe BOSS current techniques
%\ei
%Similarly to weak lensing, the galaxy clustering measurements are susceptible to
%observational and theoretical systematic effects. SIT will ensure that these    
%systematics in the key clustering measurements,such as the BAO and RSD in the   
%two-point correlation function, are are sub-dominant and don't bias the         
%recovered DE constraints. Below we briefly discuss the major systematic effects 
%anticipated for the BAO and RSD and possible mitigating techniques and          
%robustness tests.                                                               
%                                                                                
%{\it BAO systematics}: BAO measurements are expected to be relatively           
%systematics free. Analytical calculations, verified by direct numerical         
%simulations, have found the nonlinear evolution of the density field alters the 
%BAO scale by less than $0.5\%$ at the present epoch (REF by Nikhil, Martin) ,   
%and even less at the higher redshifts probed by WFIRST. A mis-calibrations in   
%the acoustic scale would affect principally the determination  of the Hubble    
%constant, not the dark energy constraints \citep{2004PhRvD..70j3523E}. Galaxy   
%formation may also result in an additional shift in the BAO scale due to        
%mismatched weighting of high and low density regions. Initial perturbative and  
%numerical work also find these shifts to be small, with the most extreme shifts 
%less than $0.5\%$ (Eisenstein 08, Tojeiro 14, Ross 15). The ``reconstruction''  
%of the density field my reduce these effects below 0.1\% (Eisenstein, Seo,      
%Spergel, Sirko 08, Padmanabhan et al. 12, Vargas-Magana, Ho 14). We will use    
%mock catalogues with known input to check that this is indeed the case for      
%WFIRST samples. We will propagate the BAO measurements all the way through the   
%science analysis pipeline and verify that the resulting DE constraints are      
%unbiased. WFIRST will have multiple galaxy types in the overlapping volume will 
%give us an additional handle on testing the robustness of measurements, since   
%the BAO signature is expected to be the same for all samples.                   
%                                                                                
%{\it RSD systematics}: The major systematic effect for the RSD measurements is  
%the theoretical modeling of various nonlinear effects (nonlinearities in real  
%to redshift space mapping, in growth of structure, and in galaxy biasing). It's 
%also possible that the velocity bias will be significant for WFIRST samples.    
%These effects cumulatively were found to be at $0.5\sigma$ (Samushia et al.     
%2014, Alam, Ho et al. 2015) in BOSS data. Possible observational systematics    
%include the effect of uncertainty in the determination of radial selection      
%function $n(z)$ which will result in apparent reduction of the signal.  Loss of 
%galaxies due to slitless-spectroscopy is also an anisotropic effect and may     
%affect the RSD significantly. SIT will study this issues using high-fidelity    
%simulations with known cosmology. Many of the systematics will be less severe   
%for WFIRST compared to current surveys. For example, the non-linear effects in  
%growth will be small due to high redshift of the sample. The $n(z)$ will        
%similarly be much better determined due to high volume and number density.    
%
%


